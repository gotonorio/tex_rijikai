%
% programming by N.Goto
%
% 理事会運営細則

\RequirePackage[l2tabu, orthodox]{nag}

\documentclass[12pt,uplatex]{jsarticle}
\和暦
\usepackage[top=10truemm,bottom=15truemm,left=25truemm,right=20truemm]{geometry}
\usepackage{calc}

% 別表をpdfファイルとして取り込むために必要。
\usepackage[dvipdfmx]{graphicx}
% 一部の文章で行間を調整するための設定。
\usepackage{setspace}
% 文章全体の行間を設定。
\renewcommand{\baselinestretch}{0.8}

% 箇条書きを()番号で表示。
% \renewcommand{\labelenumi}{(\arabic{enumi})}
\renewcommand{\labelenumi}{(\theenumi)}

% 目次を作成するレベルの深さ。
\setcounter{tocdepth}{5}

% {description}箇条書きの設定。
\renewenvironment{description}
{\begin{list}{}{
      \let\makelabel\descriptionlabel
      \setlength\labelwidth{8em}% ラベルの幅。
      \setlength\itemsep{0pt}% 行間スペース。
      \setlength\parsep{0pt}% 段落間のスペース。
      \setlength\labelsep{2pt}% ラベルと項目のスペース。
      \setlength\leftmargin{\labelwidth+\labelsep}
      }
}
{\end{list}}

% 表紙、目次のページ番号を非表示にする。
\pagestyle{empty}
% 表紙
\title{ソフィア・ガーデンズ川崎 \\  理事会運営細則}

\begin{document}
\maketitle
% 表紙に目次を入れるために改ページをコメントアウト。
% \clearpage
% 目次
\tableofcontents
\clearpage
% 本文のページ番号スタートを1に設定する。
\addtocounter{page}{-1}
% 本文ではページ番号をフッター中央に生成する。
\pagestyle{plain}

\subsubsection*{ 第1条(目的)}
\addcontentsline{toc}{subsubsection}{ 第1条(目的)}
この細則は、ソフィア・ガーデンズ川崎管理組合(以下「管理組合」という)管理規約(以下「規約」という)\
に定めた理事会の運営を適正かつ円滑に行うため、規約第71条に基づき、この細則を定める。
\subsubsection*{ 第2条(役員の選任)}
\addcontentsline{toc}{subsubsection}{ 第2条(役員の選任)}
役員は以下の通りとする。
\begin{enumerate}
\item 理事長				1名
\item 副理事長				1名
\item 会計担当理事				2名
\item 全体利用施設協議会担当理事		2名
\item 理事(理事長、副理事長、全体利用施設協議会担当理事を含む。以下同じ) 14名以内
\item 監事    2名
\end{enumerate}
\textbf{2.}理事は原則として各階毎に1名を輪番制で選任する。\\
\textbf{3.}理事長、副理事長、会計担当の選任方法は以下の通りとする。
\begin{enumerate}
\item 立候補者がいる場合はこれを優先する。
\item 立候補者がいない場合は、前任の理事による推薦と理事の話し合いで決める。
\end{enumerate}
\textbf{4.}監事は理事会の「業務監査」と「会計監査」を行うため、理事を兼ねてはならない。\\
\textbf{5.}監事は専門的知識を有する者、又は理事長経験者などから選任する。\\
\textbf{6.}理事は高齢、病気、その他のやむを得ない理由がある場合は、総会の承認を受けて輪番から免除できるものとする。\\
\subsubsection*{ 第3条(役員の任期)}
\addcontentsline{toc}{subsubsection}{ 第3条(役員の任期)}
理事の任期は、新たに就任した日の属する年度の通常総会が開かれた日の翌日から当該年度の翌々年度の通常総会の
開かれる日までの2年間とする。\\
\textbf{2.}理事は毎年その半数を改選する。\\
\textbf{3.}理事の再任は妨げないが、続けての再任は1回までとする。\\
\textbf{4.}監事の任期は基本的に第1項にある2年間とするが、続けての再任も妨げない。\\
\textbf{5.}役員が組合員の資格を失った場合には、その役員はその地位を失う。\\
\textbf{6.}監事は輪番において理事に選任されたものとしてカウントする。\\
\subsubsection*{ 第4条(理事会の開催)}
\addcontentsline{toc}{subsubsection}{ 第4条(理事会の開催)}
理事会は原則として毎月開催する。\\
\textbf{2.}理事会の議案書は開催日の3日以上前に、全ての理事、監事に配布されなければならない。\\
\textbf{3.}監事は理事会に出席して意見を述べることはできるが、議決に加わることはできない。\\
\textbf{4.}開催方法としては、会合方式以外にオンラインによる開催も可能とする。\\
\textbf{5.}議長は議事開始の前に議事録担当者を指名しなければならない。\\
\subsubsection*{ 第5条(組合員の傍聴)}
\addcontentsline{toc}{subsubsection}{ 第5条(組合員の傍聴)}
組合員から予め申し入れがあった場合、特別な事情がない限り理事会は傍聴者として会場へ入場させなければならない。
\subsubsection*{ 第6条(議事録の作成・保管)}
\addcontentsline{toc}{subsubsection}{ 第6条(議事録の作成・保管)}
議長は、書面又は電磁的記録により、議事録を作成しなければならない。\\
\textbf{2.}議事録は1週間以内に作成し、迅速に回覧または配布しなければならない。\\
\textbf{3.}電磁的議事録原本の場合、電子署名を付するものとする。ただし書面による議事録原本がある場合、
電磁的議事録には電子署名は必要としない。\\
\begin{center}
\subsubsection*{附則}
\addcontentsline{toc}{subsection}{附則}
\end{center}
\subsubsection*{第1条}
\addcontentsline{toc}{subsubsection}{ 第1条}
規約並びに本細則に定めのない事項は理事会の決定によるものとする。
\subsubsection*{第2条}
\addcontentsline{toc}{subsubsection}{ 第2条}
この細則は、令和〇〇年〇月〇日から施行するものとする。

\end{document}
